\newpage
\TOCadd{Abstract}

\noindent \textbf{Supervisory Committee}
\tpbreak
\panel

\begin{center}
  \textbf{ABSTRACT}

  With an average of more than 900 merges into the Linux kernel per
  release, many containing hundreds of commits and some containing
  thousands, maintenance of older versions of the kernel becomes nearly
  impossible. Various commercial products, such as the Android platform,
  run older versions of the kernel; due to security, performance, and
  changing hardware needs, maintainers must understand what changes
  (commits) are added to the current version of the kernel since the last
  time they inspected it in order to make the necessary patches.

  Current tools provide information about repositories through the
  directed acyclic graph (DAG) of the repository, which is helpful for
  smaller projects. However, with the scale and number of branches in the
  kernel the DAG becomes overwhelming very quickly. Furthermore, the DAG
  contains every parents of every commit, while maintainers are more
  interested in how and when a commit arrives to the official Linux
  repository.

  This paper make three contributions; a conversion from DAG to \mt, an
  implementation of a tool built on the \mt model, and a user study to
  evaluate and validate the implementation and model.
\end{center}
