\newpage
\TOCadd{Abstract}

\noindent \textbf{Supervisory Committee}
\tpbreak
\panel

\begin{center}
  \textbf{ABSTRACT}

  Understanding how commits are grouped and integrated is vital for
  understanding how and why changes were made. While direct
  visualization of the DAG of the repository will contain all
  information about the events taking place, users are unable to draw a
  strong conceptual understanding of the events taking place. The DAG
  visualization becomes overwhelming very quickly, especially at the
  scale of some of the larger repositories.

%   Between 8000 and 14000 commits are integrated into the master branch
%   of the Linux kernel per release. Due to the number of commits being
%   integrated, maintenance of older versions of the kernel becomes nearly
%   impossible.

  Current tools provide information about repositories through the
  directed acyclic graph (DAG) of the repository, which is helpful for
  smaller projects. With the scale and number of branches in the kernel
  repository, the DAG becomes overwhelming very quickly; the DAG
  visualization shows information that is not relevant to a given
  commit. Furthermore, the DAG visualization shows all parents of every
  commit, while maintainers are more interested in how and when a commit
  is integrated into the master branch.

  We propose a model designed with the intent of clearly showing how a
  commit is integrated, a tool that uses this model for summarization
  and visualizations, and an evaluation of the tool through a user
  study.
\end{center}
