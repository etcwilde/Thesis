\newpage
\TOCadd{Abstract}

\noindent \textbf{Supervisory Committee}
\tpbreak
\panel

\begin{flushleft}
  \textbf{ABSTRACT}

  Version control systems are an asset to software development, enabling
  developers to keep snapshots of the code as they work.
  Stored in the version control system is the entire history of the
  software project, rich in information about who is contributing to the
  project, when contributions are made, and the what part of the project
  they are being made.
  Presented in the right way, this information can be made invaluable in
  helping software developers further the development of the project,
  and maintainers to understand how the changes to the current version
  can be applied to older versions of projects.

  Maintainers are unable to effectively use the information stored
  within a software repository to assist with the maintanance older
  versions of that software in highly-collaborative projects.
  The Linux kernel repository is an example of such a project.
  This thesis focuses on improving visualizations of the Linux kernel
  repository, developing new visualizations that help answer questions
  about how commits are integrated into the project.
  Older versions of the kernel are used in a variety of systems where it
  is impractical to update to the current version of the kernel.
  Some of these applications include the controllers for spacecrafts,
  the
  core of mobile phones, the operating system driving internet routers,
  and as Internet-Of-Things (IOT) device firmware.
  As vulnerabilities are discovered in the kernel, they are patched in
  the current version.
  To ensure that older versions are also protected against the
  vulnerabilities, the patches applied to the current version of the
  kernel must be applied back to the older version.
  To do this, maintainers must be able to understand how the patch that
  fixed the vulnerability was integrated into the kernel so that they
  may apply it to the old version as well.

  This thesis makes four contributions:
  (1) a new tree-based model, the \mt{}, that abstracts the commits in the repository,
  (2) three visualizations that use this model,
  (3) a tool called \tool{} that uses these visualizations,
  (4) a user study
  that evaluates whether the tool is effective in helping users answer
  questions related to how commits are integrated about the Linux
  repository.


  The first contribution includes the new tree-based model, the
  algorithm that constructs the trees from the repository,
  and the evaluation of the results of the algorithm.
  the second contribution demonstrates some of the potential
  visualizations of the repository that are made possible by the model,
  and how these visualizations can be used depending on the structure of
  the tree.
  The third contribution is an application that applies the
  visualizations to the Linux kernel repository.

  The tool was able to help the participants of the study with
  understanding how commits were integrated into the Linux kernel
  repository.
  Additionally, the participants were able to summarize information
  about merges,
  including who made the most contributions,
  which file were altered the most,
  more quickly and accurately than with Gitk and the command line tools.
\end{flushleft}
