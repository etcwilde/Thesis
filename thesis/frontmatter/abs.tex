\newpage
\TOCadd{Abstract}

\noindent \textbf{Supervisory Committee}
\tpbreak
\panel

\begin{center}
  \textbf{ABSTRACT}

  The visualization of the DAG is difficult to comprehend in large
  software projects, specifically for the purpose of understanding how a
  commit is integrated and what other commits are necessary for that
  integration. Other research on repository visualization focuses on
  other aspects of the repository, not the topology.

  This thesis focuses on improving the visualization of the DAG for the
  Linux kernel repository, which is a large and highly collaborative
  software project. The repository integrates more than 8000 commits per
  release, contributed by more than 1000 active authors per release.
  Using properties found in the Linux repository, a new tree-based model
  is introduced to alleviate the issues in the DAG visualization.

  A tool is constructed using the \mt{} model as a proxy to testing the
  capabilities of the new model. A 12-participant user study is
  performed to evaluate how well the visualizations in the tool are able
  to provide comprehension.

  The findings indiacate the visualizations of the \mt{} are able to
  help users comprehend how commits are integrated more quickly and more
  accurately in the context of the Linux repository. The model should
  generalize to repositorys with a similar topology, requring that th
  repositoy must use a branch-and-merge structure, the repository must
  not have foxtrots, and cannot apply fast-forward merges.

\end{center}
