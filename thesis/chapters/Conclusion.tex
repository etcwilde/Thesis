% vim:set et sw=2 ts=4 tw=72:
\chapter{Conclusion}\label{chap:conclusion}

The visualization of the DAG of a git repository is difficult to
comprehend in large projects.
This \paper{} investigates user comprehension of the DAG
visualization in the Linux repository, and presents the design of a new
model and a tool built from the model, with the goal of assisting users
with comprehending how a commit is integrated into the master branch.
Very few tools have the explicit goal of showing the topology of the
repository. No academic tools that I am aware of attempt to do this.
Most of the non-academic tools provide a visualization of the entire
DAG, if they are able to produce a visualization at all. There is little
variance in the DAG visualizations between these tools, which leaves
room for improvement.

One major issue in understanding the DAG visualizations is the amount of
information being presented. The DAG visualization provides information
about all of the commits, but in the case of the Linux repository, the
integrating merges in the master branch work nearly independently of
each other. Only the commits that are merged together are related to
each other are relevant, while commits that are not included in that
merge are unrelated. The Linux repository adds thousands of commits per
release, but only a few of these are related to each other. 50\% of the
merges are merging at most seven commits.

The \mt{} model takes advantage of this structure, breaking the commits
into groups based on the merge into the master branch. The commits are
then organized into trees, the parent of a node is the next merge on the
way to integration, which shows the path that a commit takes to reach
the integrating merge into the master branch.  A commit may pass through
multiple merges on the way to the master branch. An algorithm is
devised, and evaluated. Through the evaluation of the algorithm, I found
some interesting events in the repository. The logs for the integrating
merges contain the number of commits being integrated, and a listing of
a subset of the commit log summaries being merged. This practice was put
into place on September 4, 2007. A foxtrot merge occurred on December
12, 2006. I identified 507 merges, of the 1537 merges made prior to this
date, that were confounded by the foxtrot.

I constructed a tool, \tool{}, around the \mt{} model. Leveraging the
model, \tool{} is able to provide simpler visualizations and summarize
additional information about the commits being merged, including the
authors involved and files modified. Through \tool{}, I am able to
further evaluate the effectiveness of the visualizations of the \mt{}
compared to the visualizations of the DAG\@.

Using \tool{}, I conducted a 12-user study with two goals. One goal is
to verify the assumption that the visualizations of the DAG are not able
to convey information about how a commit is integrated into a project.
The other goal is to compare the visualizations and summarizations from
the DAG in git and Gitk to the visualizations and summarizations from
the \mt{}. The participants were unable to accurately determine how
commits were integrated from the DAG visualization. Furthermore, the
visualizations and summarizations in \tool{} helped the participants
answer questions about a merge more accurately and more quickly. Further
information gathered from the study indicated that important information
that was present in the DAG visualizations was lost in the \mt{} model.
The order of commits with regard to each other tell the story of why a
developer is making changes. This information is lost in the \mt{}, but
is retained in the DAG visualizations.

\mt{s} are a novel means of processing git repositories to be visualized
and summarized in a more effective way. Participants in our study found
visualizations of the \mt{} to be more enjoyable for summarization tasks
than the visualizations of the DAG\@.
The visualizations of the \mt{}
model help users to more accurately summarize information about merges
more quickly.
We cannot definitively defend the original thesis statement as there
are no other tools that attempt to provide information about how
commits are integrated.
Instead, we defend the revised thesis statement.
The findings of the study show that the visualizations of the \mt{} are more effective at providing a
conceptual understanding of how a commit is integrated and what other
commits are integrated with it than tools that are currently available.
