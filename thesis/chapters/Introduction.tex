% vim:set et sw=2 ts=4 tw=72:
\chapter{Introduction}

Understanding the process in which a piece of software is developed is
crucial in the process of determining why certain changes were made, how
the changes were made, and how the surrounding code was modified to
accept those changes. Git is a distributed version control software,
designed to facilitate collaborative software development between
thousands of people working independently around the world. Git uses the
Directed Acyclic Graph (DAG) model to store modifications to code in
individual commits. The model enables git to have impressive flexibility
compared to many other version control systems, but suffers from poor
visualizations stemming from the DAG. In large projects, the
visualization of the DAG, as shown in git, does not provide an adequate
summarization of how a commit is integrated into the master branch of a
repository.


In this \paper{} \evan{dissertation, thesis or paper?}, we describe the
design of a model to alleviate this issue, called the \mt. \mt{s} are a
tree structure, rooted at the integrating merge into the master branch,
and shows the paths that commits took to be integrated. To test the
model, we construct a tool, \tool, around the repository of the Linux
kernel, which is one of the most complex open-source git repositories
available. Furthermore, git was designed by Linus Torvalds as a
replacement to Bitkeeper, the version control system that was used to
manage the development of the kernel until 2005, meaning that the git
was designed specifically for the Linux repository. Our dataset includes
commits from as far back as 2001 and as recent as 2014, though for this
\paper{} we will mostly concern ourselves with the commits contributing
to kernel versions 3.1 up to and including 3.16, or the commits that
were integrated into the kernel between October of 2011 and August of
2014.
%% TODO: More is described in the background section of the paper
