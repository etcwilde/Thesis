% vim:set et sw=2 ts=4 tw=72:
\chapter{Introduction}

A version control system records the changes being made to the files of
a project,
enabling users to view previous versions of the files,
view the individual changes made,
and restore the files back to a previous state if necessary.
The version control system also maintains a log of who made changes and
when those changes were made.
By storing this information, the version control system stores the
history of the project.
Presented in the right way, there are many opportunities to use
this information to help users understand the evolution of the software
system.
A version control system can be used in any context where file history
is needed, it is usually used in the software development process.

Git is a \define{version control system}{VCS} used by the Linux kernel
project.
Git was designed by Linus Torvalds for the Linux project as a
replacement for BitKeeper.
In order to handle the number of people
contributing to the kernel from different locations, git was designed to
be distributed.
Unlike in centralized version control systems, where users must
re-synchronize with the server, a distributed version control system
provides each user with a full first-class repository. This allows the
user to have additional flexibility, and means that a user has to
synchronize less often.
Furthermore, each user has access to the entire history of the
repository, including all branches and commits that were part of the
original repository.
Users are able to combine and re-order commits before making the changes
publicly available, which alleviates issues with synchronization between
developers.
To make it more useful to the Linux project,
git was designed to allow easy branching.
Branches allow users to work on a logically separate part of the
repository, then merge the changes into the repository once the feature
is finished or the bug is fixed.
This lets git users work independently on a feature, taking full
advantage of version control, without needing to worry about
synchronization until the feature is ready to be integrated.
To support these features, git uses a directed acyclic graph to
represent the structure of the commits in the repository.
The nodes of the graph represent the commits, containing the changes
being made and metadata about when and who made the changes.
The edges of the graph represent the parent relationship between
commits.

Visualizations of this graph are used to answer questions about the
development of the software including what changes are being made into
various branches, how the changes to the code are grouped, and who is
working with whom, among others. Maintainers use these visualizations to
understand what changes are being made to the current version of the
software in order to apply the necessary fixes to older versions of the
software to keep them secure and performing correctly. This requires
understanding how a commit is integrated into the repository, and other
commits that are merged with that commit. In large, active, software
repositories this task is not trivial.
The graph can be large and
complicated, making these visualizations difficult to understand.
The difficulties in understanding how commits are integrated into the
Linux kernel repository drives the overarching question behind this
\paper{}.

\begin{textbox}
  \textbf{Overarching Research Question:} How can we effectively
  visualize the graph of the Linux repository in a way that gives
  insight into how commits are integrated?
\end{textbox}

To answer the overarching question, this \paper{} makes four
contributions.
First, a new tree-based model, called the \mt{},
is constructed from the underlying graph of the repository.
Compared to the graph, trees are relatively easy to visualize, and there are many visualization metaphors that take advantage of different properties of trees.
The \mt{} abstracts the repository commit graph into a set of trees, each rooted at a merge into the master branch of the repository.
The leaves of the tree are the commits, and the inner nodes of the tree represent the merges leading to the integration of the commits.
Second, this \paper{} proposes three visualizations that take advantage of the \mt{} model.
Third, an implementation of the visualizations in a tool called \tool{}.
Fourth, the last contribution of this \paper{} is an evaluation of the
tool.

\begin{textbox}
  \textbf{Thesis:} Visualizations of the \mt{} are an effective means of
  visualizing the graph of the Linux repository in a way that gives
  insight into how commits are integrated.
\end{textbox}

\section{Thesis Organization}\label{sec:thesis_organization}

This \paper{} is organized as follows.
Chapter~\ref{chap:background} contains background information about the
motivation for this work and the structure of git repositories.

Chapter~\ref{chap:model} introduces the \mt{} model.
This chapter includes a description of the model, an algorithm to
convert from the DAG to a set of \mt{s}, and an evaluation of the
resulting trees built from the Linux repository graph.
At the end of the chapter is a summary of the information found in the
Linux repository, including the number of authors contributing, the
number of commits, and the average number of nodes per \mt{}.

Chapter~\ref{chap:design_and_implementation} introduces \tool{},
providing the use-cases that were being targeted.
This chapter also includes the features that were implemented into \tool{},
including its search engine, summarization tables, and tree visualizations.
More details on how the tool was implemented are included in
Chapter~\ref{chap:implementation_details}.

Chapter~\ref{chap:evaluation} is the empirical evaluation of \tool{},
and include the methodology and results of this two-part study. The
first part evaluates user comprehension of the DAG and the second part
compares visualizations and summarizations of the DAG in Gitk against
the visualizations and summarizations of the \mt{} in \tool{}.

Chapter~\ref{chap:discussion} discusses the results of the study
providing more insight on the results. The chapter includes observations
from the study, and the comments from one of the members of the study
who had worked as a release manager, and a description and algorithm for
an updated \mt{} that takes into account the comments and observations
from the study. The chapter concludes with the limitations of the work
and the future work.

Chapter~\ref{chap:conclusion} concludes that paper, reiterating the
problem addressed by this \paper{} and how it was solved.
