% vim:set et sw=2 ts=4 tw=72:
\chapter{Background}\label{chap:background}

To understand how a change is integrated into the master branch of a
repository, it is vital to understand the merges that led to the
integration of the commit, and the commits that are merged into the
master branch with the commit of interest. Maintainers of software must
sift through the commits to determine which changes being made to the
current version of the software pertain to the area of the software that
they are maintaining. Specifically, maintainers must be able to answer
two questions;

\begin{textbox}
\begin{itemize}
  \item How is a commit integrated?
  \item What other commits are related to a given commit?
\end{itemize}
\end{textbox}

Commits being integrated into a repository may pass through multiple
merges on the path to the master branch. The merges form logical
groupings of related code; furthermore, in some repository structures,
merges are used to resolve conflicts between commits. In order to apply
the changes in one commit, modifications to other parts of the codebase
may be necessary. While they are necessary to the integration of the
commit of interest, these modifications will likely be in separate
commits, and possibly merged into the branch that the commit of interest
is on. Maintainers must be able to easily find these commits, as they
will be necessary for back-porting the commit of interest into older
versions of the software.

We specifically study the Linux kernel repository. Older versions of the
Linux kernel are used in a wide variety of situations including various
Linux desktop distributions, internet of things device firmware, web
servers, spacecrafts\footnote{Linux is used heavily at SpaceX
  \url{https://lwn.net/Articles/540368/}}, and in mobile devices as the
base of the Android platform. These kernels are sometimes modified forks
of the official Linux kernel, made to be more suitable for the specific
needs of the application. Due to these application-specific
modifications, it is not feasible to update to the latest version of the
kernel for every official update. While it may not be feasible to
update, the changes being made to the official version are necessary as
they fix bugs, patch security issues, and improve performance. Due to
the sometimes critical nature of the patches being merged into the
current version of the kernel, it is necessary for maintainers working
on application-specific forks of the kernel to sift through the commits
coming into the official version, looking for the changes that may
impact the kernel they are maintaining. In Section~\ref{sec:linux}, we
will break down the repository, investigating the nature of the
repository integration and merge behaviour.

Linux uses git as the version control system, managing the distributed
nature of the kernel development and storing the changes to the kernel.
In Section~\ref{sec:git}, we will discuss the structure of git and the
challenges it poses against our two primary goals.

\section{Linux}\label{sec:linux}
\section{Git}\label{sec:git}
