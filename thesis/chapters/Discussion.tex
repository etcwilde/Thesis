% vim:set et sw=2 ts=4 tw=72:
\chapter{Discussion}\label{chap:discussion}

This chapter provides further discussion on the results and observations
from the study evaluating \tool{}. Included are observations that could
lead to improvements in the current DAG visualizations, the comments
from a release manager, and work toward applying the \mt{} model and
visualizations to other repositories.

\section{Interpreting the Results}\label{sec:interpreting_the_results}

Overall, the results indicate that \tool{} is able to improve the
correctness and accuracy of responses to various summarization tasks,
and decrease the time taken to produce the results. This doesn't come as
a surprise since the goal of the \mt{} model and the visualizations in
\tool{} are to provide better conceptual understanding and
summarizations of merges, while Gitk and DAG visualizations are designed
to show the topology of the entire repository. Since there are no other
tools for showing how a commit is integrated, and the topology of the
DAG does contain this information, the DAG visualization is used as a
proxy to show how a commit reaches the master branch.

One area of interest is the comparison of \tool{} and Gitk on
correctness in task T10, determining the modules modified in a merge.
Again, modules are not inherent to Git and are a property of the commits
in the Linux repository, the module is found in the summary of the
commit logs. In this task, the difference in the number of correct
responses between \tool{} and Gitk was not statistically significant,
and, while significant, the effect on accuracy was also small.
This is
interesting because \tool{} provided this information directly, while
users would have to look at the commit logs to determine this
information from Gitk.
Further inspection of the merges show that this
was the only task where the correct answer was in the commit that was
provided, and actually required no aggregation of the results.

Another area of interest are the time results for task T7. This is the
only task where merge size had a significant impact on the performance
of the participants. There was not a statistically significant
difference in the time taken to respond to this task between the two
tools; however, the effect size indicates that the tool has a medium
effect on the time taken to respond. This is likely due to the sample
size. In the other tasks, the responses 11 responses for both merges
were combined, effectively doubling this number, creating 22 samples.
Since there was a difference in the time taken to respond given the
merge size, the results had to be analyzed separately. 11 samples is
quite small, and is likely not enough to have a 95\% confidence in the
results.

Ultimately, the positive results are not entirely unexpected.
\tool{} is specifically designed for handling the tasks that we
presented to the participants in our study.
Ideally, we would be able to compare against another tool that is
specifically designed for the same tasks; however, one does not
currently exist.
Instead, we compare the results between \tool{}, and the tool that is
currently in use for these tasks, Gitk.

\section{Study Observations}\label{sec:study_observations}

Identifying the master branch was an issue that consistently came up
among all participants during the study. Some participants assumed that
the first line in the DAG visualization indicated the master branch,
while others assumed that the next branch tag indicated the master
branch. The DAG visualization provides no indication of which branch is
the master branch. Furthermore, the visualization in Gitk is not
consistent, branch colors and positions change between runs; identifying
the branch once does not guarantee that it is identifiable after
restarting Gitk.

Had the participants been able to easily identify the master branch, the
results from the study would likely be very different. This would be
most prominent in the summarization portion of the small merge, since
summarizing a single item is trivial, but the issue was identifying that
there was only a single item, which is currently not trivial. With more
than 25\% of the merges into Linux being single-commit merges, it is
important that users are able to identify them and understand the
changes being made within them. The structure of a single-commit tree is
identical to the structure of a flat tree, all commits are merged
directly into the master branch, passing through no other merges on the
way. Flat trees are the most common form of tree in the kernel
repository. To improve the visualization of the DAG for providing an
effective visualization for summarization and comprehension of flat
trees, it would likely be sufficient to indicate the master branch.
For the non-flat trees, a more powerful structure would likely be
necessary, such as the \mt{}.

\section{Comments From a Release Manager}\label{sec:comments_from_a_release_manager}

One of the participants in the study had worked as a release manager for
more than three years, working with both SVN and CVS repositories. The
goal of a release manager is to determine how to merge the branches of a
repository in such a way that it minimizes merge conflicts and maintains
the meaning of the underlying source code. This section contains
insights from this participant, providing comments on ways that could
improve \tool{} and the \mt{} model.

Contributors making merges need to understand more than just what merges
a commit was collected into before reaching the repository of the
contributor.
It is also important to understand the order that the
related commits were made, as the order tells the story of what the
developer was thinking as they were writing the changes.
The visualization of the \mt{} in \tool{} does not order the commits,
randomly ordering them in each level as atomic units.

This is the primary reason behind why this participant would ask to use
both tools simultaneously. \tool{} is able to help with the aggregation
of the information, and provide a better understanding of the next merge
involved in integrating this commit, but the DAG visualization in Gitk
provides the full story of the commit instead of hiding it behind a
layer of abstraction.

The comments from this participant were very insightful, and will help
to improve the \mt{} model.

\section{Threats to Validity}\label{sec:threats_to_validity}

While precautions were taken to mitigate threats to the validity, for
pragmatic reasons and other errors, some threats must be taken into
account when considering the results. These threats, the mitigation
techniques, and the steps to minimize other threats are provided in this
section.

\subsection{Internal Validity}\label{sub:internal_validity}

This section provides a summary of areas where bias may have influenced
the results of the study and the steps taken to mitigate these issues.
The biases may impact the design of the study, how the questions are
presented, and how the results are interpreted.
To ensure that this research was conducted in a way that is reasonable
and prudent, ethics was requested and approved by the University ethics
board.

The design of the study has an impact on the results.
I needed to make considerations about the merges that would be used
during the study and the tasks that the participants were working with.
The merges need to be representative of what is found in the Linux
kernel repository, but must also not be too large such that it
overwhelms the participants.
The task selection is another challenge in this study, the goal is to
limit the bias toward either tool while attempting to determine if
\tool{} has the desired effect.

The tasks themselves are biased in favour of \tool{}, as the tool is
specifically designed for answering questions surrounding conceptual
understanding of how commits are merged.
Ideally, we could compare our \tool{} with another tool designed for the
tasks specified, but such a tool does not exist.
Instead, we compare against the tool that is currently in use for
performing these tasks.
The original thesis statement must be softened to accommodate this
change though.
The visualizations of the \mt{} are more effective than the tools that
are currently employed for these types of tasks.

How the study is conducted can impact the results.
The study has two manipulated variables, the tool being used and the
number of commits being merged.
The order that people work with these will have an impact on the
results.
As a means of combating the resulting order bias, the order that commits
were analyzed, tasks performed, and tools used was randomized for each
participant.
To minimize the risk that an error is introduced by the randomization, a
script was used to generate the exact text for each experiment.
While this method proved to be very useful,
I omitted one task during the course on one study.
As a result, the information collected from this participant were
removed from the analysis and final results.

There are many merges into the master branch of the repository.
The number of commits being merged range from one to more than 6000.
The goal is to select merges that do not give an advantage to either
tool, are representative of the merges found in the repository, and are
merging a different number of commits.
The first commit studied came from a merge that was in the first
quartile.
All of the merges in the first quartile were merging only a single
commit.
One of these merges was randomly selected from the data set using a
python script to avoid any biases.
The first and second quartiles were selected to avoid overwhelming the
participants.
From prior experience, Gitk does not summarize the changes at a merge.
To summarize the changes at a merge, a developer needs to visit each
commit that is being merged and record the desired metric.
For small merges, this is feasible.
With large merges, this would become heavily burdensome on the
participants of the study.

This biases the results in favour of the tool that is better-able to
visualize small merges.
Due to this bias, I had hypothesized that Gitk and \tool{} would have
nearly identical results for the first commit, while the results would
be in favour of \tool{} for the second commits.
This was not the case, participants had a difficult time discerning
which commits were being merged in both cases.
Based on the comments and behaviours exhibited while performing the
tasks, I don't believe that increasing the number of commits will
improve the results of Gitk.

The main issue with Gitk appeared to stem from the difficulty in
determining the set of commits that belonged to a merge.
The answers provided by participants to earlier tasks were not taken
into account when evaluating the correctness or accuracies to following
tasks. The results of an incorrect answer may impact the results of the
tasks that followed. If the participant was unable to determine the
correct commits that were being merged, then none of the summarizations
would be correct, even if the response was correct given the commits
they identified. A future study could mitigate this by providing the
correct set of commits that are being merged between the conceptual and
summarization task sets to limit the propagation of errors.

\subsection{External validity}\label{sub:external_validity}

While many online git resources, Gitk, and the git command line provide
visualizations of the DAG, many participants were unfamiliar with the
DAG visualizations. Other tools than Gitk and the command line may
provide better summarizations and different visualizations, I am not
aware of any. I investigated the use of other GUI tools on the git
website, but none were able to produce visualizations for repositories
that are at the size of the Linux kernel repository, except for Gitk and
the git command line. While this may have had an effect on the results,
the tools listed on the website provide a very similar visual DAG
metaphor to the visualization in Gitk.

The participants in the study were students, some with industrial
experience. Most participants had worked with relatively few
collaborators on academic projects. Many of the participants had worked
with larger repositories while performing a research study. Even though
the participants have worked with large repositories during the course
of their research, professional developers are the target audience of
this tool, so working with professional developers would provide more
meaningful results.

\section{Limitations}\label{sec:limitations}

The model is designed with the Linux repository in mind. The viability
of \mt{s} to provide useful and accurate information relies on a few
properties of the underlying repository. The repository must use a
branch and merge structure. Some repositories, like the OCaml
repository, commit directly into the master branch. At release time, a
branch is created for the version being released. Patches to the version
are added as necessary. The release branch is never merged back into the
master branch. Since \mt{s} are designed to show how a commit is
integrated into the master branch, an \mt{} will not help with a
repository with this structure.

Repositories cannot have foxtrots. A foxtrot confounds the master
branch, making it impossible to properly determine where the integrating
merge occurs. The algorithm will continue to process repositories
containing foxtrots; however, the resulting \mt{s} will not be
meaningful.

Repositories should limit the use of fast-forward merging. The goal of
\mt{s} is to help understand how commits are grouped together, which is
done at a merge commit. Fast-forward merges splice the changes directly
into the underlying branch, hiding the fact that there was ever a
branch. The original branch information is not retrievable and will
result in many flat trees, where everything is merged directly into the
master branch, or worse, the master branch contains only individual
commits.

\section{Future Work}\label{sec:future_work}

In the evaluation of \tool{}, it was noted that some important
information is lost in the conversion from the graph to the \mt{}.
Commits usually build on the changes of it's ancestors;
the changes in one commit usually build on the changes of the commits
that came earlier in the graph.
The proposed \mt{} model does not maintain the order of the commits in
the graph, only that they share a merge.
This leaves space for additional research to build on the \mt{},
extending the model to preserve the order of the commits in the graph.
This extension would involve deciding on new visualizations, and
performing a new study.

In our model, the trees are short, but very wide, as there are usually
very few merges that a commit will pass through, but there may be many
commits that build on it before finally having the changes merged.
In the current structure, all of these commits are on the same level, as
they are merged together, which contributes to the short and wide
structure.
In the updated structure, the commits won't be on the same level, the
parent of the commit will be the next commit in the chain.
This structure leads to tall and skinny trees.
The list-tree and \rt{} tree would likely visualize these effectively.
The list-tree would only need to order the elements according to the
order of the commits, but would otherwise be left unchanged.
The \rt{} tree would become very tall, but this does not present any
major technical challenges.
The pack tree visualization is specifically designed for wide and short
trees.
If a node is the only child of it's parent, the parent node is not
drawn.
With the structure of the trees, this does not present any major issues
as there are generally multiple commits being merged by a given merge.
If the model is adjusted so that the parent of a commit is the next
commit on the path toward the master branch, this property changes.
A commit may have one or zero children, so only the first commit of a
branch will be rendered by the pack tree.

Changing the model to include information about how commits are ordered
will have an impact on how it is rendered.
Some of the visual metaphors of the tree will be impacted more severely
than others.
A re-evaluation is required to verify that the updated model and new
visualizations still fulfill the original goal, to enable people to
easily determine the merges that a commit passes through toward being
integrated, and the other commits that are integrated with it.

The work in this \paper{} addresses an issue with comprehending
visualizations of the DAG\@.
At no point is it verified that the work is applicable to the problems
faced by practitioners in industry.
While this is due to accessibility,
future work should perform a study to verify that the \mt{} model,
and the visualizations of the model are able to help solve issues in industry.
