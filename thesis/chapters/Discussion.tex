% vim:set et sw=2 ts=4 tw=72:
\chapter{Discussion}\label{cha:discussion}

This chapter provides further discussion on the results and observations
from the study evaluating \tool{}. Included are observations that could
lead to improvements in the current DAG visualizations, the comments
from a release manager, and work toward applying the \mt{} model and
visualizations to other repositories.

\section{Interpreting the Results}\label{sec:interpreting_the_results}

Overall, the results indicate that \tool{} is able to improve the
correctness and accuracy of responses to various summarization tasks,
and decrease the time taken to produce the results. This doesn't come as
a surprise since the goal of the \mt{} model and the visualizations in
\tool{} are to provide better conceptual understanding and
summarizations of merges, while Gitk and DAG visualizations are designed
to show the topology of the entire repository. Since there are no other
tools for showing how a commit is integrated, and the topology of the
DAG does contain this information, the DAG visualization is used as a
proxy to show how a commit reaches the master branch.

One area of interest is the comparison of \tool{} and Gitk on
correctness in task T10, determining the modules modified in a merge.
Again, modules are no inherent to Git and are a property of the commits
in the Linux repository, the module is found in the summary of the
commit logs. In this task, there was not a statistically significant
difference in the number of correct responses between \tool{} and Gitk,
and, while significant, the effect on accuracy was also small. This is
interesting because \tool{} provided this information directly, while
users would have to look at the commit logs to determine this
information from Gitk. Further inspection of the merges show that this
was the only task where the correct answer was in the commit that was
provided, and actually required no aggregation of the results.

Another area of interest are the time results for task T7. This is the
only task where merge size had a significant impact on the performance
of the participants. There was not a statistically significant
difference in the time taken to respond to this task between the two
tools; however, the effect size indicates that the tool has a medium
effect on the time taken to respond. This is likely due to the sample
size. In the other tasks, the responses 11 responses for both merges
were combined, effectively doubling this number, creating 22 samples.
Since there was a difference in the time taken to respond given the
merge size, the results had to be analyzed separately. 11 samples is
quite small, and is likely not enough to have a 95\% confidence in the
results.

\section{Study Observations}\label{sec:study_observations}

Identifying the master branch was an issue that consistently came up
among all participants during the study. Some participants assumed that
the first line in the DAG visualization indicated the master branch,
while others assumed that the next branch tag indicated the master
branch. The DAG visualization provides no indication of which branch is
the master branch. Furthermore, the visualization in Gitk is not
consistent, branch colors and positions change between runs; identifying
the branch once does not guarantee that it is identifiable after
restarting Gitk.

If the participants were able to easily identify the master branch, the
results from the study would likely be very different. This would be
most prominent in the summarization portion of the small merge, since
summarizing a single item is trivial, but the issue was identifying that
there was only a single item, which is currently not trivial. With more
than 25\% of the merges into Linux being single-commit merges, it is
important that users are able to identify them and understand the
changes being made within them. The structure of a single-commit tree is
identical to the structure of a flat tree, all commits are merged
directly into the master branch, passing through no other merges on the
way. Flat trees are the most common form of tree in the kernel
repository. To improve the visualization of the DAG for providing an
effective visualization for summarization and comprehension of flat
trees, it would likely be sufficient to indicate the master branch.
For the non-flat trees, a more powerful structure would likely be
necessary.
