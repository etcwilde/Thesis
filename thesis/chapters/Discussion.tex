% vim:set et sw=2 ts=4 tw=72:
\chapter{Discussion}\label{cha:discussion}

This chapter provides further discussion on the results and observations
from the study evaluating \tool{}. Included are observations that could
lead to improvements in the current DAG visualizations, the comments
from a release manager, and work toward applying the \mt{} model and
visualizations to other repositories.

\section{Interpreting the Results}\label{sec:interpreting_the_results}

Overall, the results indicate that \tool{} is able to improve the
correctness and accuracy of responses to various summarization tasks,
and decrease the time taken to produce the results. This doesn't come as
a surprise since the goal of the \mt{} model and the visualizations in
\tool{} are to provide better conceptual understanding and
summarizations of merges, while Gitk and DAG visualizations are designed
to show the topology of the entire repository. Since there are no other
tools for showing how a commit is integrated, and the topology of the
DAG does contain this information, the DAG visualization is used as a
proxy to show how a commit reaches the master branch.

One area of interest is the comparison of \tool{} and Gitk on
correctness in task T10, determining the modules modified in a merge.
Again, modules are no inherent to Git and are a property of the commits
in the Linux repository, the module is found in the summary of the
commit logs. In this task, there was not a statistically significant
difference in the number of correct responses between \tool{} and Gitk,
and, while significant, the effect on accuracy was also small. This is
interesting because \tool{} provided this information directly, while
users would have to look at the commit logs to determine this
information from Gitk. Further inspection of the merges show that this
was the only task where the correct answer was in the commit that was
provided, and actually required no aggregation of the results.

Another area of interest are the time results for task T7. This is the
only task where merge size had a significant impact on the performance
of the participants. There was not a statistically significant
difference in the time taken to respond to this task between the two
tools; however, the effect size indicates that the tool has a medium
effect on the time taken to respond. This is likely due to the sample
size. In the other tasks, the responses 11 responses for both merges
were combined, effectively doubling this number, creating 22 samples.
Since there was a difference in the time taken to respond given the
merge size, the results had to be analyzed separately. 11 samples is
quite small, and is likely not enough to have a 95\% confidence in the
results.
