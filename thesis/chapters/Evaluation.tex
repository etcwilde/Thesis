% vim:set et sw=2 ts=4 tw=72:
\chapter{Evaluation}

In June 2017, I conducted a study to quantitatively and qualitatively
evaluate the effectiveness of the visualizations and summaries presented
in \tool{} to the DAG-bed visualizations found in Gitk and the git
command-line. The study was performed in a controlled environment
running Ubuntu 14.04. Participants were allowed to use Gitk and the git
command-line tools for these tasks, I will refer to both tools as Gitk,
when working with DAG-based visualizations and summarizations. I
considered allowing participants to use any of the free tools for Linux
suggested on the git
website\footnote{\url{https://git-scm.com/download/gui/linux}}, but
after attempting to use them, I found that none were able to operate on
repositories that were as large as the Linux repository. \tool{} was
used for evaluating the \mt{-based} visualizations.

The study has two primary goals; first determining if the DAG-based
visualization is sufficient for conceptual understanding, second
comparing \tool and Gitk to determine which is more capable of providing
users with a summarization of various metrics involved with integrating
a commit into the repository.

The first part of the study are conceptual questions. These questions
are designed to test the conceptual understanding drawn from the DAG
visualization. The participants will be working with Gitk to answer
RQ\ref{RQ1}. More details about this part of the study are discussed in
Section~\ref{sec:conceptual_study}. The second part of the study involve
summarizing information about the commits that are related to a given
commit. The participants work with both Gitk and \tool{} in this part of
the study. More details are discussing in
Section~\ref{sec:summarization_study}. A third part asks for the
opinions of the participants regarding their preferences, and why they
felt that way.
