% vim:set et sw=2 ts=4 tw=72:
\chapter{Model}\label{chap:model}

This chapter introduces the \mt{} model and the algorithm used to
convert the DAG into the \mt{s}. Starting with an introduction of the
\mt{} model, the chapter includes the algorithm to convert the DAG into
\mt{s}, and an evaluation of that algorithm.

Thousands of commits are merged into the repository per year. Directly
visualizing all of the information in a meaningful way is difficult, or
potentially impossible. The merges into the master branch contain
relatively few commits, with the median number of commits being seven.
Seven nodes is trivial for visualization purposes, which indicates that
building a model that groups repository events based on the merge into
the master branch may provide more meaningful visualizations.
Furthermore, the merges into the kernel are atomic, all of the context
necessary for integrating a commit is available at the merge into the
master branch.

The \mt{} model is an abstracted representation of the DAG\@. Each \mt{}
is rooted at a merge into the master branch. The leaves of this tree are
the commits and the merges are the inner nodes. A \mt{} is built
recursively with every merge in the \mt{} merging a sub-\mt{}. Any
merges that a commit must pass through to reach the master branch become
an inner node of the tree. A \mt{} is effectively a collection of
shortest paths from a commit to the merge into the master branch,
showing the path that the commit takes. The \mt{s} also help people
understand how commits were grouped together in order to be integrated.
In this model, not only has the DAG been inverted (and simplified), but
the entire notion of parent-child relationship has been reversed. Due to
this property, the terminology will be different depending on the model
being referred to; when referring to the \mt{}, the parent is the next
node toward the merge into the master branch, or the root of the tree.
When referring to the DAG, the parent relationship is in the opposite
direction, from the root toward the branch-point. In addition to
identifying the path that a commit took to being merged, the model
enables easy aggregation of commit metadata at merges, since the
parent-child and child-parent relationship for each event is known;
commits are aware of which merge is the next merge toward integration
into the master branch, and merges are aware of which commits are being
merged.

To illustrate this model I will use a small example: assume the commits
represented in Figure~\ref{fig:repoEvents} show the sequence of events
in a repository. The sequence starts with the initial commit in the
master branch of the master repository at time $t_0$. Repository event 1
is a commit, which gets forked into a separate repository, \textit{Repo
  A}, where another commit is made, event 2. Event 5 is a merge event,
merging event 2, 3, and 4 into \textit{Repo A}. Event 5 is branched
from, commit 6 happens in the new branch, while commit 7 is added
simultaneously to the original branch in \textit{Repo A}. Events 11 and
12 are both merge events, merging changes made in \textit{Repo A} into
the master branch of the master repository. As every repository is a
first-class repository, including local copies and forks, git does not
distinguish between forked repositories and branches, and in neither
case does it explicitly record where a commit was made. In this case,
commits are performed in various repositories and branches. The DAG
representation of these events is shown in Figure~\ref{fig:repoDAG}.

\begin{figure}[htbp]
  \centering
  \resizebox{0.8\textwidth}{!}{
  \begin{tikzpicture}[auto, on grid, semithick, state/.style={circle, text=black}]
    \foreach \x in {0, 1, 2, 3, 4, 5, 6, 7}
    \draw[shift={(\x + 0.5, -0.5)}, color=black] (0cm, 4cm) -- (0pt, -0.2cm);

    \node[state, draw=chartblue] (1) {1};
    \node[state, draw=chartyellow, above right= of 1] (2) {2};
    \node[state, draw=chartmagenta, above right= 2cm and 1cm of 2] (3) {3};
    \node[state, draw=chartblue, right= 2cm of 1] (4) {4};
    \node[state, draw=chartyellow, above right=of 4] (5) {5};
    \node[state, draw=chartred, above right=of 5] (6) {6};
    \node[state, draw=chartyellow, right=of 5] (7) {7};
    \node[state, draw=chartmagenta, above right= 2cm and 1cm of 7](8) {8};
    \node[state, draw=chartyellow, right= 2cm of 7] (9) {9};
    \node[state, draw=chartyellow, right=of 9] (10) {10};
    \node[state, draw=chartblue, below right=of 9] (11) {11};
    \node[state, draw=chartblue, below right=of 10] (12) {12};

    \draw (12) edge (11) edge[chartyellow] (10);
    \draw (11) edge (4) edge[chartyellow] (9);
    \draw (10) edge[chartyellow] (9);
    \draw (9) edge[chartmagenta] (8) edge[chartred] (6)
              edge[chartyellow] (7);
    \draw (8) edge[chartmagenta] (7);
    \draw (7) edge[chartyellow] (5);
    \draw (6) edge[chartred] (5);
    \draw (5) edge[chartmagenta] (3) edge[chartyellow] (2)
              edge[chartyellow] (4);
    \draw (4) edge (1);
    \draw (3) edge[chartmagenta] (2);
    \draw (2) edge[chartyellow] (1);

    \node [draw=chartblue, below = 1.5cm of 1] (l1) [thick, minimum height=0.8cm]{Master};
    \node [draw=chartyellow, below = 1.5 cm of 4] (l2) [thick, minimum height=0.8cm]{Repo A};
    \node [draw=chartred, below = 4.5cm of 8] (l3) [thick, minimum height=0.8cm]{Branch of Repo A};
    \node [draw=chartmagenta, below = 1.5 of 12] (l4) [thick, minimum height=0.8cm]{Repo B};

        \foreach \x in {0, 1, 2, 3, 4, 5, 6, 7, 8}
    \node[shift={(\x, -0.6)}, color=black] {$t_\x$};
  \end{tikzpicture}
}
  \caption{An example sequence of events performed in different
    repositories. The horizontal axis represents time. The branches and
    repositories are aligned horizontally, and color-coded. Each commit
    points to its parent. The initial commit is at time $t_0$, and the
    head is at $t_8$.}
  \label{fig:repoEvents}
%\vspace{-3mm}
\end{figure}

The DAG does not retain information about where a commit was originally
created beyond the order of the events in the parent list. This is
necessary to provide complete flexibility to users, but at the expense
of maintaining a consistent history. Ideally, it would be possible to
completely reconstruct the information in Figure~\ref{fig:repoEvents}
from the information in Figure~\ref{fig:repoDAG}, but this may not be
possible. Instead, I focus on determining the next merge that leads
toward the integration of a commit. Depicted in
Figure~\ref{fig:repoTree}, is the first version of the \mt. It does not
completely rebuild the lost information, but is able to show the
sequence of merges that a commit follows to be integrated, and the
commits that were involved with the integration.

\begin{figure}[htbp]
  \centering
  \resizebox{0.8\textwidth}{!}{
  \begin{tikzpicture}[auto, on grid, semithick, state/.style={circle, text=black, black}]
    \node[state, black] (1) {1};
    \node[state, black, above right= of 1] (2) {2};
    \node[state, black, above right= 2cm and 1cm of 2] (3) {3};
    \node[state, black, right= 2cm of 1] (4) {4};
    \node[state, black, above right=of 4] (5) {5};
    \node[state, black, above right=of 5] (6) {6};
    \node[state, black, right=of 5] (7) {7};
    \node[state, black, above right= 2cm and 1cm of 7](8) {8};
    \node[state, black, right= 2cm of 7] (9) {9};
    \node[state, black, right=of 9] (10) {10};
    \node[state, black, below right=of 9] (11) {11};
    \node[state, black, draw=chartblue, below right=of 10] (12) {12};

    \draw (12) edge[-stealth] (11) edge[-stealth] (10);
    \draw (11) edge[-stealth] (4) edge[-stealth] (9);
    \draw (10) edge[-stealth] (9);
    \draw (9) edge[-stealth] (8) edge[-stealth] (6)
              edge[-stealth] (7);
    \draw (8) edge[-stealth] (7);
    \draw (7) edge[-stealth] (5);
    \draw (6) edge[-stealth] (5);
    \draw (5) edge[-stealth] (3) edge[-stealth] (2)
              edge[-stealth] (4);
    \draw (4) edge[-stealth] (1);
    \draw (3) edge[-stealth] (2);
    \draw (2) edge[-stealth] (1);
  \end{tikzpicture}
  }
  \caption{DAG representation of the commits represented in
    Figure~\ref{fig:repoEvents}. The DAG loses information about which
    repository the commit is performed in and through which merges it
    has passed on its way to the master branch. The DAG does not even
    distinguish the master branch from other branches.}
  \label{fig:repoDAG}
%\vspace{-3mm}
\end{figure}

\begin{figure}[htpb]
  \centering
  \resizebox{0.8\textwidth}{!}{
    \begin{tikzpicture}[auto, on grid, semithick, node distance=1cm, state/.style={circle, text=black, minimum size=7mm}]

      \node[state, draw=chartblue] (1) {1};
      \node[state, draw=chartblue, right=of 1] (4) {4};
      \node[state, draw=chartblue, right=of 4] (11) {11};
      \node[state, draw=chartblue, right=2.5cm of 11] (12) {12};

      \node[state, draw=chartyellow, above right= 1cm and 0.5cm of 11] (7) {7};
      \node[state, draw=chartyellow, above left= 1cm and 0.5cm of 11] (5) {5};
      \node[state, draw=chartyellow, right=of 7] (9) {9};
      \node[state, draw=chartyellow, left=of 5] (2) {2};
      \node[state, draw=chartmagenta, above=of 5] (3) {3};
      \node[state, draw=chartmagenta, above left=1cm and 0.5cm of 9] (6) {6};
      \node[state, draw=chartmagenta, above right=1cm and 0.5cm of 9] (8){8};
      \node[state, draw=chartyellow, above=of 12] (10) {10};

      \draw (11) edge[chartyellow, stealth-] (2) edge[chartyellow, stealth-] (5) edge[chartyellow, stealth-] (7) edge[chartyellow, stealth-] (9)
      (5) edge[chartmagenta, stealth-] (3)
      (9) edge[chartmagenta, stealth-] (6) edge[chartmagenta, stealth-] (8)
      (12) edge[chartyellow, stealth-] (10);
    \end{tikzpicture}
  }
  \caption{The \mt{s} computed for each commit in
    Figure~\ref{fig:repoDAG} showing the path that each commit takes to
    be merged into the master branch of the repository. This does not
    indicate how the events being merged are related. This figure
    retains the numerical order of the events, but the order is
    arbitrary.}
  \label{fig:repoTree}
\end{figure}

Using the depth of the node from the root of the tree, the branch
information is reconstructed. In our events, nodes 2, 5, 7, and 9, are
all on the same branch, and are merged into node 11. Nodes 5 and 9 are
merge nodes, 5 merges a single commit into the branch, and 9 merges two
nodes into the branch. In some cases, it is possible that a commit was
merged twice. In the case of node 9, it is merged into 11, though it
could also be merged into 12 through 10. The \mt{} is designed to use
the shortest distance in merges, and if two paths have the same distance
from the master branch, use the shortest distance through time.

\section{Algorithm}
\label{sec:algorithm}

Computing the \mt from a DAG for any repository may not be possible;
however, certain features of the development process of Linux make it
feasible to compute the \mt for the Linux repository. First, the master
branch of Linux is maintained by Linus Torvalds, and only Linus has
write access to it. This assertion is verified by
German~\cite{German2015}. The heuristic developed from this information
is presented in Algorithm~\ref{fig:alg}. In short, the algorithm first
identifies the commits made directly to the master branch, whereafter it
recursively determines the shortest path, using the DAG, from each
commit to the master branch using the inverted DAG\@.

\begin{algorithm}
        \caption{Computing the \mt of Linux from the DAG}\label{fig:alg}
        \begin{algorithmic}[1]
                \Function{ComputeMergeTree}{DAG}: tree
                % \State {\# Compute the tree from the DAG of Linux repository.}
                % \State {\# Returns $Tree$, a graph containing every commit }
                % \State {\# in DAG with the path it followed to master.}
                \State $head \gets \textit{Head of master of git repository}$
                \State $master \gets \textit{traverse DAG from head using }$
                \State \quad\quad\quad\quad $\textit{first parent until reaching root}$
                \State $nodes(Tree) \gets nodes(DAG)$
                \State \Function{MergeAtMaster}{cid}
                \State {\# Returns $(depth, merge, next)$}
                \State {\# Helper function}
                \State {\# Compute the closest merge into master, }
                \State {\# setting the children on the way to master.}
                \If {\textit{cid in master}}
                \State \Return $(0, cid, \varnothing)$
                \EndIf
                \State {$d \gets \infty$}
                \State {\# Traverse the inverted DAG}
                \For{$c \in children(cid, DAG)$}
                \State $(d_c, merge_c, next_c) \gets MergeAtMaster(c)$
                \If {$IsMerge(c)$}
                \State $fp \gets FindFirstParent(c)$
                \If {$fp \neq cid$}
                \State $d_c \gets d_c + 1$
                \State $next_c \gets c$
                \EndIf
                \EndIf
                \State {\# Find the shortest path}
                \If {$d_c < d$}
                \State $(d, m, next) \gets (d_c, merge_c, next_c)$
                \ElsIf{ $d_c = d$ }

                \State {\# Use the time as a tie-breaker}
                \If {$ cTime(merge_c) < cTime(m) $}
                \State $(m, next) \gets (merge_c, next_c)$
                \EndIf
                \EndIf
                \EndFor
                \State {\# $c$ is the commit that follows $cid$ on it's way to master}
                \State add edge $(cid, next)$ to $Tree$
                \State \Return $(d, m, next)$
                \EndFunction

                \State {\# Compute the distance for each commit discarding result}
                \For{$c \in nodes(DAG)$}
                \State $MergeAtMaster(c)$
                \EndFor
                \State \Return $Tree$
                \EndFunction
        \end{algorithmic}
\end{algorithm}

The algorithm is broken into two phases. The first is determining which
repository events are on the master branch. This is done by traversing
the first parent from the master branch reference to the commit that has
no parents. The second phase is encompassed by the function
$MergeAtMaster$ which determines, for each commit, which merge the
commit is merged at, the depth (as variable $d$ in the algorithm), and
the next merge on the path to the master branch. The function
$MergeAtMaster$ has two parts, the first for determining the depth, from
the master branch, that the repository event is at. The second phase
determines the merge into the master branch, and the next merge on the
way to the master branch. The distance is by shortest path, staying as
close to the master branch as possible. If there is a tie between two
paths, the path that merges into the master branch sooner is taken.

An example is used to demonstrate the behaviour of the algorithm,
computing the merge at commit 5 in Figure~\ref{fig:repoEvents}.
$MergeAtMaster$, recurses along the children of the nodes it visits.
Eventually every child of every node along the path will be visited at
least once. Without loss of generality, suppose that the path recursed
along is from node 5 to 6, 9, 10, and finally 12.

The depth for each, except 12 (a merge into the master branch), is
initialized to infinity, the merge into master is blank, and the next
merge is blank. Merges into master trivially have a distance of 0 from
the master branch, and it merges itself into the master branch. The
recursion at 12 returns the triple $(0, \varnothing, 12)$ to the call
from 10. 12 is a merge commit and 10 is not the first parent, so the
temporary depth, $d_c$, is incremented to 1 and the temporary next
merge, $next_c$, is changed to 12. 1 is less than infinity, so the depth
is set to 1, the merge to 12, and the next to 12. This returns the
triple $(1, 12, 12)$ to the call from 9. 9 is the first parent of 10, so
no changes are made to the temporary variables.

The call to 9 recurses to the second child, 11. 11 is a merge into the
master so it returns $(0, \varnothing, 11)$ to the call from 9. 9 is not
the first parent of 11, so the $d_c$ is incremented to 1 and $next_c$ is
changed to 11. The distances $d_c$ and $d$ are the same, so time is used
to break the tie. 12 was merged after 11, so 11 replaces 12 as the
merge into the master branch for 9, as well as being the next merge. The
call for 9 returns the triple $(1, 11, 11)$ to the call for 6. 6 is not
the first parent of 9, so $d_c$ is incremented and $next_c$ is changed
to 9, as 9 merges 6. 2 is less than infinity, so the $d$ is changed to
2, the merge to 11, and the next merge to 9. The call to 6 returns the
triple $(2, 11, 9)$ to the call for 5.

The call for 5 recurses on the second child of 5, calling on 7, which
calls 8, and then 9. 9 can continue, but if the implementation of the
algorithm uses memoization, the call to 9 can immediately return the
triple $(1, 11, 11)$ to the call for 8, and avoid an exponential
runtime. 8 is not the first parent of 8, so $d_c$ is incremented to 2
and $next_c$ is changed to 9. 2 is less than infinity, so $d$ is changed
to 2, $merge$ to 11, and $next$ to 9. The call to 8 returns the triple
$(2, 11, 9)$ to the call for 7, which recurses on the second child of 7,
9. 9 returns the triple $(1, 11, 11)$. 7 is the first parent of 9, so
the depth is not incremented. $d_c$ is less than $d$, so $d$ is changed
to 1, $m$ to 11, and $next$ to 11, returning $(1, 11, 11)$ to the call
for 5. $d_c$ is less than $d$, so $d$ is changed to 1, $m$ to 11, and
$next$ to 11. There are no other children, so the function halts.

\section{Algorithm Evaluation}
\label{sec:algorithm_evaluation}

The trees must be validated to ensure that they are an accurate
representation of the events occurring in the repository. Evaluation
poses some issues, as there is no easy way to accurately gather this
information directly from the DAG\@. Further inspection of the merges
provides some insight; Linus Torvalds adds useful information about the
content of the merges. For each branch being merged, Linus includes
either the first 20 commit titles and the total number of commits being
merged (see Figure~\ref{fig:sampleMerge} for an example), or the list of
titles if 20 or fewer commits are being merged from the branch.

\begin{figure}[htpb]
  \centering
\begin{textbox}
  \begin{verbatim}
Merge: 8cbd84f fd8aa2c
Author: Linus Torvalds <torvalds@linux-foundation.org>
Date:   Tue Aug 10 15:38:19 2010 -0700

Merge branch 'for-linus' of git://neil.brown.name/md

* 'for-linus' of git://neil.brown.name/md: (24 commits)
md: clean up do_md_stop
[... edited for the sake of space]
md: split out md_rdev_init
md: be more careful setting MD_CHANGE_CLEAN
md/raid5: ensure we create a unique name for kmem_cache...
...
  \end{verbatim}
\end{textbox}
  \caption{Example of how merges record a subset of commits being merged. The
                commit only shows the first 20 one-line summaries messages for the 24
                non-merge commits it merged. The ending ``\ldots'' is part of the log
                and represents that other commits were merged.}
  \label{fig:sampleMerge}
\end{figure}

This information is used to verify that the results from the database
match the information in the merge logs, and further investigate the
merges that don't. If the \mt{} in our database has the correct
number of commits being merged, and the titles listed in the log appear
in the \mt{}, then the commits and \mt{} have been correctly
identified. A scripted approach was used to extract the information from
both the git logs and  the database. The commit hash, creation date, and
number of commits merged are extracted from each merge log.

This information is gathered from the logs using two commands;\\
\verb|git log --merges --author='Torvalds' v3.16| to collect the merges
made by Linus, and \verb|git log --format="%H" --first-parent --merges  v3.16| to
collect merges along the master branch.

15099 merges were collected from the database, and 15306 merges from the
logs using this technique. There was an overlap of 14836 merges between
the two, with 470 merges from the logs that were not in the database,
and 260 merges that were in the database that were not in the logs, or a
total of 15566 merges.

The main issue behind the disparity between the number commits collected
from the logs and the commits from the database are the technique that
they were collected. All information for merges made by Linus since the
merge tag \emph{v3.16} were collected from the logs, regardless of
whether they are in the master branch. The set of repository events
along the master branch is determined by following the first-parent
starting at the merge tag, which is on the master branch, and recording
the events visited. Merges that are not on the master branch are not
removed from the dataset. Only merges collected for the database were
identified to be along the master branch. Focusing only on the merges in
the master branch, the difference between the two sets decreases
dramatically. There were 14670 merges along the master branch that were
shared by both the logs and by our database, 2 merges,
\emph{186051d70444742bf1c2bc0257dd4696a3df66e3} and
\emph{5170a3b24a9141e2349a3420448743b7c68f2223}, from the logs that were
not in the database, and 426 merges in the database that were not in the
logs. Examining the two merges from the logs, the master branch
disappears into a group of commits, which are later merged. This is
confusing, as Linus is the only one with direct merge access into the
master branch of the repository. The likely reason for this is that
these commits were merged in a fast-forward merge, which does not insert
a merge, and instead splices the commits directly into the branch.

Conversely, many of the merges and commits that were in our database but
not found in the logs were due to the fast-forward merges. The commits
were detected as being part of the master branch when inserting the
commits into the database, but were not collected by our git log query
since they were not authored by Linus Torvalds.

Beyond investigating why merges show up in either the database or logs,
further analysis requires that the merges are present in both the
database and the merge log. There are 14198 merges with these
properties.

The results of the evaluation are summarized as follows:

\begin{itemize}
  \item

    Five merges did not have matching commit counts between the database
    and the logs. Upon further investigation, four merges had
    incorrectly formatted logs. The fifth merge,
    \emph{42a579a0f960081cd16fc945036e4780c3ad3202}, is a \foxtrot{}
    merge. One commit is on the first-parent of the merge, and is
    therefore not detected when building the \mt{}, but is included
    in the merge log.

  \item

    The heuristic worked correctly until September 4, 2007, the earliest
    date that could be verified. Before this date, merge logs did not
    include a summary of the commits being merged, making it impossible
    to verify. Manual inspection indicates that the heuristic worked
    correctly for these commits, until December 12, 2006 where a
    \foxtrot{} merge occurs.

  \item

    There is 1 merge after September 4, 2007 that does not have recorded
    commit logs. This is due to incorrect formatting. If it were
    correctly formatted, it would report having 15 commits integrated,
    which is consistent with the results in the database.

  \item

    There were 1537 merges made by Linus prior to December 12, 2006. 507
    of these merges are not on the master branch, indicating that a
    foxtrot occurred. Upon further investigation, Dave Jones merges
    29 commits from ranging from October 15, 2006 to December 12, 2006
    into the master branch. Due to the \foxtrot{}, this merges 4708
    repository events, including 199 merges by Linus Torvalds.

  \item

    Another foxtrot merge by Jeff Garzik on June 26, 2005 merges a
    total of 2349 repository events, 105 of which are merges by Linus
    Torvalds. The intent was to merge 10 commits into the master branch.

  \item

    77 Merges were made by Linus into non-master branches after
    September 4, 2007. These merges were made into
    \emph{3f17ea6dea8ba5668873afa54628a91aaa3fb1c0}, which is
    extraordinarily large containing 7217 repository events, 6809 of
    which are commits.

\end{itemize}

There were 12837 merges after September 4, 2007. With the exception of
the five merges, four with errors, and one as part of a \foxtrot{}, all
merges were correctly identified. The 835 merges between December 12,
2006, and September 4, 2007, appear to be correct, but cannot easily be
verified. Of the 1542 commits prior to December 12, 2006, roughly 512
merges appear to be confounded by \foxtrot{s}.
