% vim:set et sw=2 ts=4 tw=72:
\chapter{Implementation Details}

The implementation is broken into two parts, the tool and the database.
The database uses postgresql, an advanced opensource database management
system. The tool backend uses Python and the Flask framework.
Both the database and the \tool{} backend are containerized using
Docker. This facilitates easy migration between the development
environment and the server where the system is running. Both components
are running Alpine 3.5, a simple, light, and secure distribution of
Linux designed to run as the base of Docker containers. Containers are
similar to virtual machines, but easier to work with. Instead of using
virtualization of the hardware, Docker containers run on top of the host
kernel, making use of \textit{cgroups} to allocate resources to the
container. Since the container is not a full virtual machine, it only
needs to contain the information required to run the service. There are
no device drivers, or even a kernel. A container image can be exported
and imported from the development environment directly into the
production environment.

This chapter contains details on the design of the database, the
extraction of the data, and the implementation of the tool.

\section{Database}\label{sec:database}

The database is made of seven tables;

\begin{itemize}
  \item

    Baseline contains the commits that have been integrated into the
    master branch of the repository.

  \item

    Commits contains the metadata for each commit, including the author,
    committer, the date the commit was authored, and the date the commit
    was committed, as well as the associated patch.

  \item

    Filesmod contains the information regarding which files were changed,
    and how many lines were added and removed from each.

  \item

    Logs contains the subject and full log message for every commit.

  \item

    Releases contains information regarding which commits represent the
    split between versions of the kernel. It contains the version name,
    the commit, the previous version, the commit for the previous
    version, the previous candidate and commit, and whether the version
    is a candidate.

  \item

    Search contains information necessary for the search engine. While
    this table is not necessary for the operation of the tool, it is
    used as a cache and index, which improves the performance of the
    search engine greatly. It contains the search term vector and the
    associated commit hash.

  \item

    PathToMerge contains the actual merge-tree structures, with
    references pointing from a given commit to the next merge on the way
    to integration.

\end{itemize}

\evantodo{In the appendix, actually spell out the schema of each table
  in the database}

\subsection{Full Text Search}\label{sub:full_text_search}


We use the full text search built into PostgreSQL to enable easy
searching for commits in the database. While it is possible to compute
the text search vectors for every query, it is faster to pre-compute the
vectors for each commit. The search vectors are pre-computed and stored
in the search table, creating a map between search terms and commit
hash. The search vectors include terms from the commit hash, log
subject, full log message, commit author name, commit and authorship
date, and list of files that were modified in the commit.

The search attributes are given different weight when calculating the
search rank. The attribute weights are listed in
Table~\ref{tab:search_attribute_rank}. We do not normalize against
document length when computing the rank. The ranking system is designed
to associate a higher weight with attributes of more importance. One
drawback of using full text search is that terms must be in verbatim of
the term used in the attribute. This has two negative effects, the first
is that typos will drop commits that are otherwise relevant, and second,
the entire term must be entered. Trigraphs are a possibility for working
around this. Trigraphs break the words in the attribute into groups of
three. Using this on the commit hash, for example, would allow a user to
perform a substring search, typing only a portion of the commit hash.
Unfortunately, the Trigraphs are slower to compute, to search, and
consume more memory, and therefore were infeasible for this project.

\begin{table}[htpb]
  \centering
  \caption{Search Attribute Ranking}
  \label{tab:search_attribute_rank}
  \begin{tabular}{cc}
    \toprule
    Attribute       & Weight\\
    \midrule
    Commit Hash     & A \\
    Log Subject     & A \\
    Log Full Text   & C \\
    Author Name     & A \\
    Commit Date     & B \\
    Authorship Date & B \\
    File names      & B \\
    \bottomrule
  \end{tabular}
\end{table}

Postgres applies stop-word elimination when constructing the search
vectors; stop words are terms that are common in a given language and
won't help when discriminating items. These terms include words like
``and'', ``the'', and ``a''. We use the default English stop-words built
into Postgres version 9.6.2.
